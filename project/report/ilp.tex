\section{Integer Linear Program} \label{s:ilp}
The general problem of $P|prec|\sum_j w_jC_j$ can be expressed as an optimization problem, which minimizes $\sum_j w_jC_j$, subject to job durations, the precedence constraints and the machine capacity constraints,
\begin{align}
p_j \leq C_j& \:\:\:\: \forall j, \label{eq:ptime}\\
C_i + p_j\leq C_j&  \:\:\:\: \forall (i,j) \in E, \label{eq:prec}\\
\sum_{j | C_j - p_j \leq t \leq C_j} 1 \leq m& \:\:\:\: \forall t \in \mathcal{T}, \label{eq:macap}
\end{align}
where $\mathcal{T}$ is the time horizon of scheduling these jobs and can be upper bounded by $\sum_j p_j$. 

However, the constraint \eqref{eq:macap} is not written in the standard form. To do this, for each job we introduce a time-indexed variable $I_j^t = \{0, 1\}$ to express the constraints. It indicates a single point on the time line at which the job completes its execution, therefore
\begin{align}
\sum_t I_j^t = 1 \:\:\:\: \forall j. 
\end{align}

Notice that $\sum_t tI_j^t = C_j$ captures the time stamp of the job completion event occurring at $t$, and the objective becomes to minimize $\sum_j w_j \sum_t tI_j^t$. We can then translate the precedence constraint \eqref{eq:prec} and capacity constraint \eqref{eq:macap} as 
\begin{align}
\sum_t t I^t_i + p_j \leq \sum_t tI^t_j& \:\:\:\: \forall (i,j) \in E, \label{eq:prec_ti}\\
\sum_{j} \sum_{\tau=t}^{t + p_j} I^\tau_j \leq m& \:\:\:\: \forall t \in \mathcal{T}, \label{eq:macap_ti}
\end{align}
where the second summation in \eqref{eq:macap_ti} effectively acts as a sweeping line to count the overlapping of job execution at each time stamp over the horizon. 

This framework can be extended to include other variants of the problem. For instance, jobs may be subject to release time constraint, which can be expressed as $\sum_t tI^t_j \geq s_j + p_j$, where $s_j$ is the release time of job $j$. Also, each job may also request for multiple types of resources (e.g., CPU and RAM), then the capacity constraints \eqref{eq:macap_ti} can be extended to $\sum_{j} \sum_{\tau=t}^{t + p_j} I^\tau_j r^q_j \leq m^q, \forall q\forall t$, where $q$ is each dimension of the resource and $m^q$ is the capacity of the resource in the type $q$. In fact, multi-resource precedence constrained job\footnote{The processing time $p_j$ can actually be estimated in production clusters as a large volume of jobs is consist of recurrent jobs, where the statistics are relatively stable and can be sampled from historical data.} scheduling is suitable for practical datacenter clusters~\cite{graphene}. A production system may favor the weighed completion time of the last jobs of each job-group from different applications~\cite{tetris}, which can be included in this framework by assigning $0$ weights to all non-ending jobs.

The main motivation for having this ILP formulation is to implement a baseline to compare against various approximation algorithms in the following sections. 
