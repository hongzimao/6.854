\section{Problem description} \label{s:problem}
We consider a set $N$ of $n$ jobs, and $m$ identical parallel machines. Each job
$j$ has a nonnegative processing time $p_j$, defined on discrete time stamps $t = 0, 1, 2,
..., T$. Each machine can only be assigned with a single job at a time. We assume no
preemption, meaning each job needs to run continuously once it is assigned to a
machine. The precedence constraints are encoded in a directed acyclic graph
(DAG) $D=(N,E)$: for each precedence-constrained job pair $(i,j) \in E$ job $j$
can not start its execution until job $i$ is finished. Each job $j$ is also associated with a
weight $w_j$. The objective is to minimize the weighted completion time $C_j$
over all jobs $\sum_j w_j C_j$. This general problem is denoted as  $P|prec|\sum_j w_jC_j$
using the notation in~\cite{graham1979optimization}. Also, this model can be tailored to include the makespan objective $P|prec|C_{max}$, which measures the completion time of the last job. We can transform the problem by introducing an artificial job preceded by all other jobs, assigning unit weight and zero processing time to this final job, and leaving all other jobs zero weights. It is well known that the general version of the problem is NP-hard. In particular, the major difficulty stems from the precedence constraint, in which even the special case of single machine $1|prec|\sum_j w_jC_j$ is NP-hard \cite{lenstra1978complexity}.