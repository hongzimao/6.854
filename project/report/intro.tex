\section{Introduction} \label{s:intro}
Scheduling problems with precedence constraints are among the most difficult
problems in the topic of machine and job scheduling, especially in designing
approximation algorithms. We first present an Integer Linear Program (ILP) for
the general form $P|prec|\sum_j w_jC_j$ of the problem (\S\ref{s:ilp}), which
minimizes the average weighted job completion time on multiple machines under
precedence constraints. Then in \S\ref{s:lpr} we review in details of a
4-approximation algorithm based on LP
relaxation~\cite{queyranne2006approximation}, whose simple form of reading a
list order from LP midpoints actually yields a general framework of precedence
constraints. In the context of single machine scheduling $1|prec|\sum_j w_jC_j$
(\S\ref{s:bism}), interestingly a simple 0-1 bipartite special instance
effectively captures the inherent
difficulty~\cite{woeginger2003approximability}. We review \cite{schulz2011near}
for its new study with a probabilistic lens on the classic 0-1 bipartite
instance, which shows a near optimal solution with large probability in balanced
case. Finally in \S\ref{s:impl}, we implement the ILP and surveyed approximation
algorithms to compare their performance and run time. It verifies the statement
and observations in~\cite{schulz2011near} with a small scale experiment. The
implementations can also be flexibly extended to express the variants of the
problem, such as to also capture the release time constraints, or
multi-dimensional resources constraints.



